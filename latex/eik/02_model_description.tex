\chapter{Yield Table Model description}


\section{Growth}
Hans Jansen uses 7.5 in his $db_{at}$ calculation

wheel

This chapter describes the original model used to create the Dutch yield tables of 2018, which has resulted in an extensive collection of yield tables \parencite{jansenOpbrengsttabellenNederland20182018}. The main source for this summary is the detailed description of the model for the yield tables of Douglas fir \parencite{Jansen2016}. 

According to Jansen, a yield table is a model that describes stand development in time which usually consists of three sub-models:
\begin{itemize}[nosep] 
    \item A model for the height development in time which enables the creation of site indices;
    \item A model for basal area increment in time or relative to the height, with which the productivity of a stand can be forecasted;
    \item A model that defines the separate thinning intensities.
\end{itemize}

\section{Height growth}

The height increment is predicted depending on the age of the stand and the site class. Two separate formulas are used; one for a top height up to 7 metres and another formula for stands with a top height above 7 metres. Five distinct site indices have been distinguished, with the Roman numeral 'I' denoting the best growing class and 'V' indices the slowest. Each site index is represented by the h$_{70}$ of the class, indicating the top height ($h_{top}$) of a stand at the age of 70 years. This specific age was chosen mostly arbitrarily, with the reasoning that it approaches the average rotation length used in Dutch forestry practise today. The height increment is determined by calculating the difference of the current height and the next years height and depends on theoretical maximum top heigh for the chosen site index.

% top heigth nog eens controleren! klopt misschien niet? in engels wel dominante hoogte?
% hoogteberekening loopt via de S waarde. Wat is die? Dat is de maximumhoogte

\section{Basal area increment}
Like the calculation of height growth, the approach to calculating basal area increment and the diameter differs for stands with a top height of seven meters and below and stands which are taller. For stands below the threshold the mean quadratic diameter before thinning ($d_{bt}$) is calculated first, depending on the top height and initial stand number. This value is then used to calculate the basal area before thinning($G_{bt}$) again depending on initial stem number. From the second year the basal area increment($Ic_G$) can be calculated by subtracting the previous basal area value from the current one.

When a stands top height exceeds 7 meters, the basal area increment is calculated depending on the height increment of that year and the thinning prescription.

%De oplossing met Formule 70 is hier model C genoemd. De waarden van de gevonden parameters staan in Tabel 27. Voor het traject tot een opperhoogte van 7 m kan volstaan worden door met behulp van Formule 46 de diameter voor dunning te bepalen, en daaruit het boomgrondvlak voor dunning en ten slotte het grondvlak voor dunning in m2/ha.  In Paragraaf 6.3 zal een procedure worden besproken om de kwaliteit van het totale opbrengstmodel te toetsten. Er zal blijken dat het C model het best voorspelt. Daarna worden in Paragrafen 6.4 en 6.5 nog andere eigenschappen van de te ontwikkelen opbrengsttabel getoetst. Op grond van deze toetsen is het model voor de grondvlakbijgroei verschillende malen bijgesteld.

\section{Thinning}

The effect of thinning on stand development is modelled using the thinning grade, which is based on the Hart-Becking Spacing Index or S-percentage of Hart \parencite{Hart1928}. This index is the ratio of the available space for a tree after thinning divided by the top height (equation \ref{eq:hart}).

\begin{equation}
S\% = \frac{a_{at}}{h_{top}} \cdot 100 = \frac{100}{h_{top}} \cdot \sqrt{\frac{10.000}{N_{at}} \cdot \frac{2}{\sqrt{3}}} \approx \frac{10.745,7}{h_{top} \cdot \sqrt{N_{at}}}
\label{eq:hart}
\end{equation}

\begin{flushleft}
\hspace*{2em}\textbf{where} \\
\hspace*{2em}
\begin{tabular}{ll}
  $S\%$ & Hart-Becking Spacing Index \\
  $a_{at}$ & Average tree distance after thinning \\
  $h_{top}$ & Top height
\end{tabular}
\end{flushleft}


The thinning grade (Tgr) is calculated from the S\% using equation \ref{eq:tgr}

\begin{equation}
Tgr = \frac{S\%-10}{3} 
\label{eq:tgr}
\end{equation}

\begin{flushleft}
\hspace*{2em}\textbf{where} \\
\hspace*{2em}
\begin{tabular}{rl}
  $Tgr$ & Thinning grade \\
  $S\%$ & Hart-Becking Spacing Index
\end{tabular}
\end{flushleft}

\begin{table}[h]
    \centering
    %\begin{tabular}{|c|c|c|}
    \begin{tabular}{c c c}
        %\hline
        \textbf{Tgr} & \textbf{S\%} & \textbf{Description} \\
        \hline
        1 & 13 & ongedund \\
        %\hline
        2 & 16 & zwakke laagdunningRow  \\
        %\hline
        3 & 19 & matige laagdunning\\
        %\hline
        4 & 22 & sterke laagdunning \\
        %\hline
        5 & 25 & zeer sterke laagdunning \\
        %\hline
        6 & 28 & open stand \\
        %\hline
    \end{tabular}
    \caption{thinning grades (Tgrc) and the related Hart-Becking spacing index}
    \label{tab:placeholder_label}
\end{table}


\section S\text{\%} of Hart


\section{Model flow}


