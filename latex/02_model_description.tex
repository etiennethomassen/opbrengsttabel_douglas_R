% 2. Intial model description
\section{Yield Table Model description}
This chapter contains a short description of the Dutch yield tables as they have been developed by Hans Jansen, which has resulted in an extensive collection of yield tables \cite{jansenOpbrengsttabellenNederland20182018}. The main source for this summary is the detailed description of the model for the yield tables of Douglas fir \cite{Jansen2016}. According to Jansen, a yield table is a model that describes stand development in time which usually consists of three sub-models:
\begin{itemize}[nosep]
    \item A model for the height development in time which enables the creation of site indices;
    \item A model for basal area increment in time or relative to the height, with which the productivity of a stand can be forecasted;
    \item A model that defines the separate thinning intensities.
\end{itemize}

\subsection{Height growth}
The heigth increment is predicted depending on age and site class. In order to predict the height increment as closely as possible to reality, the height increment is calculated with two separate formula; one for trees below 7 meters in height and another formula for trees starting 7m.
De hoogteontwikkeling wordt voorspelt afhankelijk van de leeftijd en boniteit(groeiklasse). Om de hoogteontwikkeling van het model zo goed mogelijk overeen te laten komen met de werkelijkheid wordt de hoogteontwikkeling tot 7 meter opperhoogte met een andere formule berekend dan vanaf 7 meter. Er worden vijf boniteiten onderscheiden, waarbij I de best groeiende is en V de langzaamst groeiende boniteit. De hoogtes worden gepresenteerd met een bijpassende h70; de opperhoogte van de opstand bij een leeftijd van 70 jaar. De leeftijd van 70 jaar is arbitrair gekozen met het idee dat dit ongeveer overeenkomt met een hedendaagse in de praktijk gebruikelijke omloop.

\subsection{Basal area increment}
The diameter is modeled depending on the opperhoogte, age, site index, and growing space.


\subsection(S percent)
\subsection(Thinninggrade)

\subsection{Dunning}
